\documentclass[a4paper,12pt]{article}

\usepackage{amsmath}
\usepackage{amssymb}
\usepackage{textcomp}
\usepackage{inputenc}
\usepackage{fontenc}
\usepackage{graphicx}
\usepackage[a4paper]{geometry}

%\renewcommand{\thesection}{\Roman{section}} 
%\renewcommand{\thesubsection}{\thesection.\Roman{subsection}}
\makeatletter
\newcommand*{\rom}[1]{\expandafter\@slowromancap\romannumeral #1@}
\makeatother

\geometry{top=1.0in, bottom=1.0in, left=1.5in, right=1.5in} 

\title{Regression Analysis}
\author{}
\date{1 Oct 2012}
\begin{document}
\maketitle
%\begin{center}
%\textbf{Regression Analysis} - 27 July 2012\\ 
%\end{center}

\noindent (\rom{1}) \textsc{Problem} \\
\hspace*{5mm} The problem is as follows: Given a set of $N$ data points \[\mathcal{D}=\{(x_1,y_1),(x_2,y_2),\ldots,(x_N,y_N)\},\] find a function that best approximates the underlying function $f(x_i) = y_i$ which relates the data. The nature of the underlying function that defines the data or the details of how the data was generated is usually unknown. These data points might be totally random or may be related by a characteristic function. \\

\noindent (\rom{2}) \textsc{Theory} \\
\hspace*{5mm} In the absence of the underlying function, one could treat the data as a linear combination of some set of functions.
\begin{align}
y &=  \sum_{i=1}^M w_i \phi_i(x) \\
  &=  \bar{w}^T \bar{\phi}(x), 
\end{align}
where $M$ is the number of terms, $\bar{w} = [w_1 w_2 \ldots w_M]^T$ (superscript $T$ refers to the matrix transpose), and $\bar{\phi}(x) = [\phi_1(x) \phi_2(x) \ldots \phi_M(x)]^T$. Here, $w_i$ is the weight corresponding to the function $\phi_i$. \\

\noindent (\rom{2}.\rom{1}) \textsc{Orthogonal Functions} \\
\hspace*{5mm} Consider two functions $\phi_1(x)$ and $\phi_2(x)$ defined over the range $[a,b]$. The \textit{inner product} of these functions is defined as
\[ <\phi_1,\phi_2> = \int_{a}^{b} \phi_1(x) \phi_2(x) \]
$\phi_1$ and $\phi_2$ are said to be \textit{orthogonal} if $<\phi_1,\phi_2> = 0$. If there is a set of functions $\{\phi_1,\phi_2,\ldots,\phi_M\}$ defined over a range, then these functions form an orthogonal set if the inner product of any pair of these orthogonal functions $<\phi_i,\phi_j> = 0$ for $i \neq j$. \\

\noindent (\rom{2}.\rom{2}) \textsc{Fourier Series} \\
\hspace*{5mm} A Fourier series is a decomposition of a periodic function into a sum of infinite sine/cosine functions. Any periodic function $f(x)$ with fundamental period $T$ can be represented using Fourier series as:
\[ f(x) = a_0 + \sum_{n=1}^{\infty} \left( a_n cos \frac{2 n\pi x}{T} + b_n sin \frac{2 n \pi x}{T} \right) \]
The Fourier coefficients $a_n$ and $b_n$ can be determined from the following integrals:
\begin{align*}
 a_0 &= \frac{1}{T} \int_{0}^{T} f(x) dx \\
 a_n &= \frac{2}{T} \int_{0}^{T} f(x) cos \frac{2 n\pi x}{T} dx \\ 
 b_n &= \frac{2}{T} \int_{0}^{T} f(x) sin \frac{2 n\pi x}{T} dx 
\end{align*}

\textsc{Fourier series representation of a Sawtooth wave}
Consider the sawtooth function below with a peak value of 1.

The functional form of this wave is \[ f(x) = \frac{x}{T} \] The sawtooth wave is a periodic function (time period $T$) and hence can be decomposed using Fourier series representation. The components of the Fourier series are therefore given by:
\begin{align}
 a_0 &= \frac{1}{T} \int_{0}^{T} \frac{x}{T} dx  \notag\\
     &= \frac{1}{2} \\
 a_n &= \frac{2}{T} \int_{0}^{T} \frac{x}{T} cos \left( \frac{2n\pi x}{T} \right) dx  \notag\\
     &= 0 \\
 b_n &= \frac{2}{T} \int_{0}^{T} \frac{x}{T} sin \left( \frac{2n\pi x}{T} \right) dx  \notag\\
     &= -\frac{1}{n\pi}      
\end{align}
The Fourier series is therefore given by
\begin{equation}
 f(x) = \frac{1}{2} - \frac{1}{\pi} \sum_{n=1}^{\infty}\frac{1}{n}sin\left(\frac{2n\pi x}{T}\right)
\end{equation}

\textsc{Fourier series representation of a Square wave}
Consider the square wave shown below.


% \hspace*{5mm} As a particular case, the set of functions $\{\phi_i(x)\}$ belong to a set of orthogonal basis collection of functions. Any two functions in this orthogonal basis set satisy the following two criteria:-
% \[ 
%   \int\limits_{a}^{b}\phi_i(x)\phi_j(x) dx =  
%   \begin{cases}
%     1, & \text{if } i = j, \\
%     0, & \text{otherwise}.
%   \end{cases}
% \]

\noindent $(1)$ is a \textit{linear model} for regression. Approximating the output values ${y_n}$ using the linear model results in an error. Minimizing this error forms the backbone of this approach. The error in approximation is given by $(y_n-\hat{y_n})^2$ where $\hat{y_n}$ is the estimated $y$ value. The combined error for all $N$ data points can then be written as
\begin{equation}
 \mathcal{E} = \sum_{n=1}^N (y_n-\hat{y_n})^2
\end{equation} 
The error given a $M$ and $\bar{w}$ is 
\begin{equation}
 \mathcal{E}(\bar{w}) = \sum_{n=1}^N (y_n-\sum_{i=1}^M w_i \phi_i(x_n))^2
\end{equation} 

The optimal set of weights are those that minimize $\mathcal{E}(\bar{w})$
\[ \bar{w}^* = \displaystyle\operatorname*{argmin}_{\bar{w}} \mathcal{E}(\bar{w}) \]

Differentiating $\mathcal{E}(\bar{w})$ with respect to $\bar{w}$, we have
\begin{align*}
 \frac{d}{d\bar{w}}\mathcal{E}(\bar{w}) &= \frac{d}{d\bar{w}}\sum_{n=1}^N (y_n-\bar{w}^T \bar{\phi}(x_n))^2 \\
					 &= 2 \sum_{n=1}^N (y_n- \bar{\phi}(x_n)^T\bar{w})\bar{\phi}(x_n) 
\end{align*}
Now,
\[ \frac{d}{d\bar{w}}\mathcal{E}(\bar{w}) = 0 \Rightarrow \sum_{n=1}^N (y_n - \bar{\phi}(x_n)^T\bar{w})\bar{\phi}(x_n) = 0 \]
\begin{align}
\therefore \sum_{n=1}^N y_n \bar{\phi}(x_n) &= \sum_{n=1}^N \bar{\phi}(x_n)^T \bar{w} \bar{\phi}(x_n)
\end{align}
If $
    \bar{y} = \left[ 
		      \begin{array}{c}
		       y_1 \\ y_2 \\ . \\ . \\ . \\ y_N
		      \end{array}
	      \right]
   $
 and $
      \Phi = \left[
		      \begin{array}{c}
		       \bar{\phi}(x_1)^T \\ \bar{\phi}(x_2)^T \\ . \\ . \\. \\ \bar{\phi}(x_N)^T
		      \end{array}
	     \right]_{N\times M} 
     $,
then $(5)$ can be expressed as
\begin{align}
(\Phi^T \Phi) \bar{w} &= \Phi^T \bar{y} \notag \\
	      \bar{w} &= (\Phi^T \Phi)^{-1} \Phi^T \bar{y}
\end{align}

\noindent \textsc{Experiment} \\
Tests are done by simulating data from two functions in particular, namely, the \texttt{sawtooth} and \texttt{square} functions. Each of these functions can be represented using an infinite Fourier series representation.  \\

\noindent \textsc{Data Generation} 
\begin{itemize}
\item \textit{Generating X's:} The range from which the $x$ values need to be generated is defined at runtime via command-line arguments using the parameters \texttt{low} and \texttt{high}. Number of samples (\texttt{nsamples}) is also specified at runtime. Using a random data generator, these $x$ values are obtained. 

\item Corresponding to a particular function that is also specified at runtime, the function values $f(x)$ for the respective $x$'s are computed.

\item \textit{Generating Y's:} To the previously generated $f(x)$ values, some amount of \textit{Gaussian noise} is added to account for any errors in the actual experiment conducted. 
\[ y = f(x) + \epsilon \quad \textrm{and } \quad \epsilon \sim \mathcal{N}(\mu,\sigma) \]



\end{itemize}

\end{document}























